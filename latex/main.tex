\documentclass{article}
\usepackage{graphicx} % Required for inserting images

\title{Diffusion-limited aggregation\\[10pt]
    \large Programmazione di SIstemi Embedded e Multicore}
\author{Simone Di Cesare}
\date{Febbraio 2024}

\begin{document}

\maketitle

\section{Introduzione}
Il Diffusion-limited aggregation (DLA) è un processo nel quale, partendo da uno spazio N-dimensionale popolato da K particelle, si formano cristalli. Il mio progetto si concentra sull'ottimizzazione di questo processo, utilizzando tecnologie come CUDA e OpenMP per ridurre i tempi di generazione dei cristalli.

\section{Il Progetto}
Il progetto è stato sviluppato sul sistema operativo Arch Linux, installato su un HP Omen del 2018, facendo uso del compilatore GCC per le versioni seriale e OpenMP e NVCC per la versione CUDA.
\subsection{La Simulazione}
La simulazione è stata implementata considerando le seguenti caratteristiche:
\begin{itemize}
    \item Spazio bidimensionale.
    \item Assenza di collisioni con i bordi dello spazio.
    \item Generazione di un singolo cristallo iniziale.
    \item Uso di un seed statico per le particelle.
\end{itemize}
\subsection{Funzionamento della Simulazione}
Il funzionamento della simulazione è suddiviso in una serie di passaggi chiave che vengono eseguiti per ogni particella per un numero di iterazioni specificato.
\begin{enumerate}
    \item \textbf{Lettura dei dati in ingresso}:
    \begin{itemize}
        \item Vengono letti i seguenti parametri in ingresso:
        \begin{itemize}
            \item Dimensioni della griglia (larghezza X altezza).
            \item Numero di particelle.
            \item Numero di iterazioni da svolgere.
            \item Posizione iniziale del primo cristallo (X, Y).
            \item (Facoltativo) File di output per la mappa (default: crystal.txt).
        \end{itemize}
    \end{itemize}
    \item \textbf{Inizializzazione della griglia}:
    \begin{itemize}
        \item La griglia viene inizializzata in un vettore ad una dimensione.
    \end{itemize}
    \item \textbf{Inizializzazione delle particelle}:
    \begin{itemize}
        \item Viene assegnato un seed univoco a ciascuna particella (seed = index * 4).
        \item Viene assegnata una posizione iniziale randomica a ciascuna particella, basata sul suo seed.
    \end{itemize}
    \item \textbf{Esecuzione della Simulazione}:
    \begin{itemize}
        \item Per ogni particella e per ogni iterazione:
            \begin{enumerate}
                \item \textbf{Controllo dei vicini}:
                    \begin{itemize}
                        \item Se la particella ha un vicino già cristallizzato, la particella stessa cristallizza e si ferma.
                    \end{itemize}
                \item \textbf{Movimento della Particella}:
                    \begin{itemize}
                        \item La particella si muove in una direzione casuale, determinata dal suo seed.
                    \end{itemize}
                \item \textbf{Normalizzazione della Posizione}:
                    \begin{itemize}
                        \item Se la particella esce dalla griglia, viene riportata all'interno seguendo delle regole di normalizzazione.
                    \end{itemize}
            \end{enumerate}
    \end{itemize}
\end{enumerate}

\section{Performance}
Come facilmente deducibile dal funzionamento della simulazione, una versione seriale dell'algoritmo sarà inefficente tanto quanto sono grandi il numero di particelle e/o di iterazioni della simulazione.
Ad esempio, se eseguiamo la simulazione con questi parametri di ingresso:
\begin{enumerate}
    \item Larghezza Griglia: 10000
    \item Altezza Griglia: 10000
    \item Numero di Particelle: 32768
    \item Numero Iterazioni: 1000000
    \item Posizione del primo Cristallo: (99, 99) 
\end{enumerate}
Osserviamo come il tempo di esecuzione totale è pari a 2566.17 secondi (quasi 43 minuti) suddivisi in:
\begin{itemize}
    \item 40.10 millisecondi in inizializzazzione.
    \item 2557.53 secondi in simulazione.
    \item Restante in stampa e pulizia della memoria.
\end{itemize}
Per ovviare a queste alte tempistiche di simulazione, si propongono due strategie utilizzate per abbassare i tempi di simulazione.
\subsection{OpenMP}
OpenMP è un'interfaccia di programmazione che permette, tramite delle direttive pragma, di scrivere codice che può essere eseguito simultaneamente su più core della CPU.\\
Tramite questo meccanismo, ci è possibile suddividere le particelle tra più thread che verranno poi eseguiti simultaneamente, mantenendo invariato il funzionamento della simulazione ma aumentando l'efficenza.
Nello specifico, la direttiva che permette una migliore performance è:\\
\textbf{\#   pragma omp parallel for schedule(static, 4)}\\
Applicata sia per l'inizializzazione delle particelle, sia per il movimento di esse ad ogni iterazione, questa direttiva va ad eseguire in parallelo il for che va a muovere le particelle, assegnando staticamente le iterazioni del ciclo in chunck di 4.
Quest'utlimo parametro \textbf{schedule(static, 4)} serve ad evitare il false sharing; Il numero 4 deriva dalla comune lunghezza delle linee di cache ad oggi, ossia 64 byte, diviso le dimensioni della struttura dati contenente le particelle, ossia 16 byte. Perciò, possono entrare al più 4 particelle (64/8) per linea di cache, e questo comporta ad assegnare chunk da 4 iterazioni per thread.\\
Eseguendo la simulazione con i parametri precedenti otteniamo un tempo totale di 919.59 secodni (15 minuti), suddiviso in:
\begin{itemize}
    \item 57.71 millisecondi in inizializzazzione.
    \item 910.37 secondi in simulazione.
    \item Il restante in stampa e pulizia della memoria.
\end{itemize}
Notiamo come, a patto di un piccolo aumento nel tempo di inizializzazzione, andiamo drasticamente a ridurre i tempi di simulazione, ottenendo uno Speedup di 2.79, con un'efficenza di 0.23 (considerando i 12 core del computer).
\subsection{CUDA}
CUDA è una piattaforma di calcolo parallelo sviluppato da NVIDIA, con lo scopo di sfruttare la GPU per elaborare dati in parallelo.\\
Per ottimizzare la simulazione vengono lanciati tanti thread quante particelle, suddivisi in blocchi contenenti 256 thread, i quali si occupano di inizializzare e muovere le singole particelle.\\
Notiamo come, utilizzando gli stessi parametri della simulazione seriale, otteniamo un tempo totale d'esecuzione di 175.79 secondi (2.93 minuti), suddiviso in:
\begin{itemize}
    \item 180.57 millisecondi in inizializzazzione.
    \item 163.45 secondi in simulazione.
    \item Il restante in stampa e pulizia della memoria.
\end{itemize}
Come per OpenMP abbiamo un aumento della durata di inizializzazzione, ma andiamo a ridurre di circa 15 volte il tempo d'esecuzione totale della simulazione.

\section{Correttezza}
Per mantenere invariato il risultato della simulazione per le varie versioni, viene utilizzato un seed statico per ogni particella. Il seed viene calcolato a partire dall'indice
di quest'ultima, seguendo la formula \textbf{seed = index * 4}. Ogni qualvolta che bisogna estrarre un numero randomico, viene utilizzato il seed della particella e aggiornato con il nuovo valore generato.
In questo modo, per ogni simulazione non ci sarà variazione alcuna nel movimento o nel posizionamento delle particelle a prescindere dalla versione del programma lanciato.
\end{document}
